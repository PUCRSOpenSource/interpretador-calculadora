\documentclass[12pt]{article}
\usepackage[brazilian]{babel}
\usepackage[utf8]{inputenc}
\usepackage[T1]{fontenc}
\usepackage{listings}
\usepackage{color}
\usepackage{morefloats}
\usepackage{amsmath}
\usepackage{float}
\usepackage{graphicx}

\restylefloat{table}

\definecolor{dkgreen}{rgb}{0,0.6,0}
\definecolor{gray}{rgb}{0.5,0.5,0.5}
\definecolor{mauve}{rgb}{0.58,0,0.82}
\sloppy

\title{Compiladores\\Faculdade de Informática - PUCRS}

\author{
Diego Jornada \footnote{diego.jornada@acad.pucrs.br - 11204078} \\ 
Isadora Kurtz \footnote{isadora.kurtz@acad.pucrs.br - 13108765} \\
Marina Barros \footnote{marina.barros@acad.pucrs.br - 11104974} \\
}


\begin{document}

\maketitle

\begin{abstract}

O trabalho aborda o projeto do semestre da cadeira de Compiladores, cujo objetivo é criar um interpretador para uma calculadora seguindo as descrições apresentadas na especificação. 


\end{abstract}



\section{Introdução}
Dentro do escopo da disciplina, a primeira parte do trabalho, visa a implementação de um analisador léxico, especificação sintática utilizando o gerador BYACC. 

Basicamente, o analisador léxico verifica um código fonte e verifica se uma gama de símbolos é válida ou não. Quando uma palavra é reconhecida como válida, ela então compõe um token léxico. Já o analisador sintático processa o conjunto de tokens e verifica se é possível formar uma entrada válida através da construção de uma árvore de derivação. A gramática fornecida para o gerador de analisadores léxicos é uma gramática livre de contexto.

\section{Analisador Léxico}


\section{Analisador Sintático}


\end{document}