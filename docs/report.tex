\documentclass[12pt]{article}
\usepackage[brazilian]{babel}
\usepackage[utf8]{inputenc}
\usepackage[T1]{fontenc}
\usepackage{listings}
\usepackage{color}
\usepackage{morefloats}
\usepackage{amsmath}
\usepackage{float}
\usepackage{graphicx}

\restylefloat{table}

\definecolor{dkgreen}{rgb}{0,0.6,0}
\definecolor{gray}{rgb}{0.5,0.5,0.5}
\definecolor{mauve}{rgb}{0.58,0,0.82}
\sloppy

\title{Compiladores\\Faculdade de Informática - PUCRS}

\author{
Diego Jornada \footnote{diego.jornada@acad.pucrs.br - 11204078} \\ 
Isadora Kurtz \footnote{isadora.caterina@acad.pucrs.br - 13104445} \\
Marina Barros \footnote{marina.barros@acad.pucrs.br - 11104974} \\
}


\begin{document}

\maketitle

\begin{abstract}

O trabalho aborda o projeto do semestre da cadeira de Compiladores, cujo objetivo é criar um interpretador para uma calculadora seguindo as descrições apresentadas na especificação. 


\end{abstract}



\section{Introdução}

Dentro do escopo da disciplina o trabalho visa a implementação de uma calculadora $bash$ $calculator$, que precisa respeitar as seguintes especificações: 

\begin{itemize}
    \item  A calculadora trabalha com os tipos numéricos: (double) e lógico (boolean) 
    \item  Vai ter apenas 3 casas decimais
    \item  Funções recursivas são permitidas
    \item  Existirão 3 modos básicos de operação: Imediato, Atribuição, Operação
\end{itemize}
\newpage
Vai possuir as seguintes variáveis de Controle:

\begin{itemize}
    \item  Help: Auxílio básico sobre o uso da calculadora
    \item  Load: Carregar as declarações de funções
    \item  Save: Grava conteúdo da tabela de funções
    \item  Show: Dados armazenados na tabela de funções
    \item  Show All: Dados armazenados na tabela de funções
\end{itemize}

Assim como, vão existir as seguintes operadores válidos: Aritméticos, Relacionais, Lógicos, Atribuição, Condicional, e de Sequencia.
Também será implementado os comandos de: Seleção, Repetição, Impressão.O trabalho foi implementado usando a linguagem Phyton, usando a biblioteca ply (Python Lex-Yacc).

\section{Analisador Léxico}

No analisador léxico, foram feitas as seguintes implementações:

\begin{verbatim}
     reserved = {
            '#help'     : 'HELP',
            '#load'     : 'LOAD',
            '#save'     : 'SAVE',
            '#show'     : 'SHOW',
            '#show_all' : 'SHOW_ALL',
            '#print'    : 'PRINT',
            'if'        : 'IF',
            'else'      : 'ELSE',
            'while'     : 'WHILE',
            'for'       : 'FOR',
            'define'    : 'DEFINE'
            }
\end{verbatim}



\section{Analisador Sintático}

O analisador sintático 

\end{document}